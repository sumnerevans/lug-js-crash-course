\documentclass{lug}

\usepackage{csquotes}
\MakeOuterQuote{"}
\usepackage{listings}

\title{JavaScript Crash Course}
\author{Sumner Evans}

\begin{document}

\begin{frame}
    \frametitle{JavaScript is \textbf{\textit{NOT}} Java \footnotemark[1]}

    \begin{itemize}[<+->]
        \item JavaScript was written was created in 10 days in May 1995 by Brendan Eich.
        \item JavaScript was originally called Mocha and was renamed to LiveScript before being
            renamed again to JavaScript.
        \item Why \textbf{Java}Script? Because Java happened to be popular then (that was before
            people realized how awful it is) and JavaScript looks syntactically similar at a glance.
        \item JavaScript is standardized\footnotemark[2] by Ecma International and there have been a
            number of ECMAScript versions. The latest is ECMAScript 6, but it is not supported by
            most browsers, including Firefox which only has partial support.
    \end{itemize}

    \footnotetext[1]{Lots of this information is from:
    \url{https://www.w3.org/community/webed/wiki/A_Short_History_of_JavaScript}}
    \uncover<4>{\footnotetext[2]{JavaScript standards aren't actually that standard.}}
\end{frame}

\begin{frame}
    \frametitle{JavaScript Syntax Overview: Variables}

    \begin{itemize}
        \item \texttt{var name;} - creates variable \texttt{name} with value \texttt{undefined}.
        \item \texttt{var name = 'Sumner';} - creates variable \texttt{name} with value
            \texttt{"Sumner"}.
        \item Variables have types, but they can change.
    \end{itemize}

%    \begin{listing}
%        var name;
%        var age = 3;
%        var gender = "male";
%        var a = 3,
%            b = 4;
%
%    \end{listing}
\end{frame}

\end{document}
