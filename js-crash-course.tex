\documentclass{lug}

\usepackage{csquotes}
\MakeOuterQuote{"}
\usepackage{listings}

\title{JavaScript Crash Course}
\author{Sumner Evans}

\begin{document}

\begin{frame}
    \frametitle{JavaScript is \textbf{\textit{NOT}} Java \footnotemark[1]}

    \begin{itemize}[<+->]
        \item JavaScript was written was created in 10 days in May 1995 by Brendan Eich.
        \item JavaScript was originally called Mocha and was renamed to LiveScript before being
            renamed again to JavaScript.
        \item Why \textbf{Java}Script? Because Java happened to be popular then (that was before
            people realized how awful it is) and JavaScript looks syntactically similar at a glance.
        \item JavaScript is standardized\footnotemark[2] by Ecma International and there have been a
            number of ECMAScript versions. The latest is ECMAScript 6, but it is not supported by
            most browsers, including Firefox which only has partial support.
    \end{itemize}

    \footnotetext[1]{Lots of this slide's information is from:
    \href{https://www.w3.org/community/webed/wiki/A_Short_History_of_JavaScript}{w3.org/community/webed/wiki/A_Short_History_of_JavaScript}}
    \uncover<4>{\footnotetext[2]{JavaScript standards aren't actually that standard.}}
\end{frame}

\begin{frame}
    \frametitle{Variables}

    \begin{itemize}
        \item \texttt{var name;} - creates variable \texttt{name} with value \texttt{undefined}.
        \item \texttt{var name = `Sumner';} - creates variable \texttt{name} with value
            \texttt{"Sumner"}.
        \item Variables have types, but they can change.
    \end{itemize}

%    \begin{listing}
%        var name;
%        var age = 3;
%        var gender = "male";
%        var a = 3,
%            b = 4;
%
%    \end{listing}
\end{frame}

\begin{frame}
    \frametitle{Scope}

    There are two scopes in JavaScript: global and function.\footnotemark[1]\\

    Variables are \textit{hoisted} to the top of a function.

    \footnotetext[1]{In ES6, variables declared with \texttt{let} are actually block scope.}
\end{frame}

\begin{frame}
    \frametitle{Arrays}

    JavaScript arrays are basically vectors.
\end{frame}

\begin{frame}
    \frametitle{Objects}

    JavaScript objects are basically maps.
\end{frame}

\begin{frame}
    \frametitle{Functions}

    Functions are just objects with two special properties: a context (scope) and the function code.
\end{frame}

\begin{frame}
    \frametitle{Functions: Closure}

\end{frame}

\begin{frame}
    \frametitle{Inheritance}
    JavaScript is Pseudoclassical.

\end{frame}

\begin{frame}
    \frametitle{Truthy, Falsy and \texttt{==} vs \texttt{===}}

    JavaScript has the notion of being \textit{truthy} and \textit{falsy}.

    JavaScript has two equality operators:
    \begin{itemize}
        \item \texttt{==} compares without checking variable type.
        \item \texttt{===} compares and checks variable type.
    \end{itemize}
\end{frame}

\begin{frame}
    \frametitle{Pitfalls}

\end{frame}

\begin{frame}
    \frametitle{Libraries}

    There are \textbf{lots} of JavaScript libraries. One of the most widely used is jQuery
    (\href{http://jquery.com/}{jquery.com}). It has good documentation and is really good for DOM
    manipulation.\\

    I also have a few nice prototype overloads in a repository at
    \href{https://github.com/sumnerevans/js-utils}{github.com/sumnerevans/js-utils} (MIT License).

\end{frame}

\begin{frame}
    \frametitle{DOM Manipulation}

    The \textit{Document Object Model} is an API used by JavaScript to interact with the elements of
    an HTML document.\footnotemark[1]

    \footnotetext[1]{\href{https://en.wikipedia.org/wiki/Document_Object_Model}{en.wikipedia.org/wiki/Document_Object_Model}}
\end{frame}

\begin{frame}
    \frametitle{Additional Resources}

    A lot of this presentation was based off of \textit{JavaScript: The Good Parts} by Douglas
    Crockford. This is an essential read for anyone interested in learning JavaScript for anything
    more than writing a few simple scripts.\\

    MDN is the best resource for JavaScript documentation
    (\href{https://developer.mozilla.org/en-US/}{developer.mozilla.org}).
\end{frame}

\end{document}
