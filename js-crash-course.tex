\documentclass{lug}

\usepackage{csquotes}
\usepackage{listings}

\MakeOuterQuote{"}
\setbeamerfont{footnote}{size=\Tiny}
\lstset{
    basicstyle=\small,
    breaklines=true
}

\title{JavaScript Crash Course}
\author{Jonathan Sumner Evans}

\begin{document}

\begin{frame}
    \frametitle{JavaScript is \textbf{\textit{NOT}} Java \footnotemark[1]}

    \begin{itemize}[<+->]
        \item JavaScript was written was created in 10 days in May 1995 by Brendan Eich.
        \item JavaScript was originally called Mocha and was renamed to LiveScript before being
            renamed again to JavaScript.
        \item Why \textbf{Java}Script? Because Java happened to be popular then (that was before
            people realized how awful it is) and JavaScript looks syntactically similar at a glance.
        \item JavaScript is standardized\footnotemark[2] by Ecma International and there have been a
            number of ECMAScript versions. The latest is ECMAScript 6, but it is not fully supported
            by any browsers, including Firefox which only has partial support.
    \end{itemize}

    \footnotetext[1]{Lots of this slide's information is from:
    \url{https://www.w3.org/community/webed/wiki/A_Short_History_of_JavaScript}}
    \uncover<4>{\footnotetext[2]{JavaScript standards aren't actually that standard.}}
\end{frame}

% TODO: Put the high level stuff here

\begin{frame}
    \frametitle{Variables}

    JavaScript is an \textbf{untyped} language. I don't know what that means and I don't
    think that Brendan did either when he wrote the language.\\

    Variables are declared using the \texttt{var} keyword\footnotemark[1]. \\

    \textbf{Examples:}

    \begin{itemize}
        \item \texttt{var name;} - creates variable \texttt{name} of type \texttt{undefined}.
        \item \texttt{var name = `Sumner';} - you can initialize a variable when you declare it.
    \end{itemize}

    \footnotetext[1]{Sometimes you don't need to use \texttt{var}.}
\end{frame}

\begin{frame}
    \frametitle{Types\footnotemark[1]}

    JavaScript has six primitive types:

    \begin{itemize}
        \item Boolean (\texttt{true} or \texttt{false})
        \item Null
        \item Undefined (yes, this is a type)
        \item Number (can be a number between $-(2^{53} - 1)$ and $2^{53} - 1$, \texttt{NaN},
            \texttt{-Infinity}, or \texttt{Infinity}).
        \item String (single or double quotes declares a string literal\footnotemark[2])
        \item Symbol (new in ECMAScript 6)
    \end{itemize}

    ~\\
    Everything else is an object.

    \footnotetext[1]{Info on this slide from:
    \url{https://developer.mozilla.org/en-US/docs/Web/JavaScript/Data_structures}}
    \footnotetext[2]{Single quotes is recommended by Douglas Crockford because HTML normally uses
        double quotes and to avoid conflicts when manipulating DOM objects, single quotes should be
    used.}
\end{frame}

\begin{frame}[fragile]
    \frametitle{Scope I}

    There are two scopes in JavaScript: global and function.\footnotemark[1]\\

    Variables are \textit{hoisted} to the top of the function they are declared in. Thus, the
    following is entirely valid.

    \begin{lstlisting}
    function() {
        b = 5;
        console.log(b); // logs 5
        var b = 3
        console.log(b); // logs 3
    }
    \end{lstlisting}

    Thus, it is recommended that you declare all of your variables at the top of your
    functions (one exception to this rule is counter variables).

    \footnotetext[1]{In ES6, variables declared with \texttt{let} are actually block scope.}
\end{frame}

\begin{frame}[fragile]
    \frametitle{Scope II}

    Variables declared outside of a function are automatically in the global scope.\\

    Variables declared within a function \textit{without} the \texttt{var} keyword are also in the
    global scope.

    \begin{lstlisting}
    var a = 2;
    (function() {
        b = 3
    })(); // this creates and invokes the function immediately

    console.log(a); // logs 2
    console.log(b); // logs 3
    \end{lstlisting}

    Because your code could coexist with other people's code, on the same HTML page, it is
    recommended that you reduce your \textit{global footprint} by creating only a few global objects
    and then putting everything into that.
\end{frame}

\begin{frame}
    \frametitle{Objects}

    JavaScript objects are basically maps.

    % Talk about the prototype chain @ a high level
\end{frame}

\begin{frame}
    \frametitle{Functions}

    Functions are just objects with two special properties: a context (scope) and the function code.
\end{frame}

\begin{frame}
    \frametitle{Functions: Closure}

\end{frame}

\begin{frame}
    \frametitle{Functions: Callback}
\end{frame}

\begin{frame}[fragile]
    \frametitle{Arrays}

    JavaScript arrays are basically vectors.\\

    \textbf{Example:}

    \begin{lstlisting}
    var arr = [1, 'a', {}, [], true];
    arr[0] = 'not a number';
    arr.push('this is basically a vector');
    console.log(arr);
    \end{lstlisting}

    \textbf{Output:}

    \begin{lstlisting}
    [ 'not a number', 'a', {}, [], true, 'this is basically a vector' ]
    \end{lstlisting}

    \textit{Note that the elements of an array do not have to be the same type.}

\end{frame}


\begin{frame}
    \frametitle{Inheritance}
    JavaScript is Pseudoclassical.

\end{frame}

\begin{frame}
    \frametitle{Truthy, Falsy and \texttt{==} vs \texttt{===}}

    JavaScript has the notion of being \textit{truthy} and \textit{falsy}.

    JavaScript has two equality operators:
    \begin{itemize}
        \item \texttt{==} compares without checking variable type.
        \item \texttt{===} compares and checks variable type.
    \end{itemize}
\end{frame}

\begin{frame}
    \frametitle{Pitfalls}

\end{frame}

\begin{frame}
    \frametitle{DOM Manipulation}

    The \textit{Document Object Model} is an API used by JavaScript to interact with the elements of
    an HTML document.\footnotemark[1]

    \footnotetext[1]{\url{https://en.wikipedia.org/wiki/Document_Object_Model}}
\end{frame}

\begin{frame}
    \frametitle{Libraries}

    There are \textbf{lots} of JavaScript libraries. One of the most widely used is jQuery
    (\url{http://jquery.com/}). It has good documentation and is really good for DOM
    manipulation.\\

    I also have a few nice prototype overloads in a repository at
    \url{https://github.com/sumnerevans/js-utils} (MIT License).

\end{frame}

\begin{frame}
    \frametitle{Additional Resources}

    A lot of this presentation was based off of \textit{JavaScript: The Good Parts} by Douglas
    Crockford. This is an essential read for anyone interested in learning JavaScript for anything
    more than writing a few simple scripts.\\

    MDN is the best resource for JavaScript documentation
    (\url{https://developer.mozilla.org/en-US/}).
\end{frame}

\end{document}
