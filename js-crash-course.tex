\documentclass{lug}

\usepackage{csquotes}
\usepackage{listings}
\usepackage{minted}

\MakeOuterQuote{"}
\setbeamerfont{footnote}{size=\Tiny}
\lstset{
    basicstyle=\small,
    breaklines=true
}

\title{JavaScript Crash Course}
\author{Sumner Evans and Sam Sartor}

\begin{document}

\begin{frame}
    \frametitle{JavaScript is \textbf{\textit{NOT}} Java \footnotemark[1]}

    \begin{itemize}[<+->]
        \item JavaScript was written was created in 10 days in May 1995 by Brendan Eich.
        \item JavaScript was originally called Mocha and was renamed to LiveScript before being
            renamed again to JavaScript.
        \item Why \textbf{Java}Script? Because Java happened to be popular then (that was before
            people realized how much Java sucks in a browser) and JavaScript looks syntactically
            similar at a glance.
        \item JavaScript is standardized\footnotemark[2] by Ecma International and there have been a
            number of ECMAScript versions. The latest is ECMAScript 6, but it is not fully supported
            by any browsers, including Firefox which only has partial support.
    \end{itemize}

    \footnotetext[1]{Lots of this slide's information is from:
    \url{https://www.w3.org/community/webed/wiki/A_Short_History_of_JavaScript}}
    \uncover<4>{\footnotetext[2]{JavaScript standards aren't actually that standard.}}
\end{frame}

\begin{frame}
    \frametitle{Objects}

    \begin{itemize}[<+->]
        \item Everything is either a primitive or an object.
        \item Objects in JavaScript are mutable keyed collections.
        \item JavaScript is \textit{pseudoclassical}.
        \item JavaScript uses \textit{prototypes} for inheritance.
        \item There is no such thing as a \textit{class} in JavaScript.\footnotemark[1]
    \end{itemize}

    \footnotetext[1]{ECMAScript 6 added support for classes, but JavaScript classes are just
    wrappers around the underlying prototype-based structure.}
\end{frame}

\begin{frame}
    \frametitle{Objects: Inheritance and the Prototype Chain}

    \begin{itemize}[<+->]
        \item Every JavaScript object is linked to a \textit{prototype}. Objects inherit the
            properties from their prototypes.

        \item Object literals inherit from \texttt{Object.prototype} which is defined by the
            JavaScript language.

        \item You can set the prototype of an object to another object by calling
            \texttt{myObj.prototype = otherObj;}

        \item Since the prototype of an object is itself an object, the prototype will have a
            prototype.

        \item The prototype relationship is a dynamic relationship. If a property is added to the
            prototype, it is automatically visible to all objects based on that prototype.
    \end{itemize}

\end{frame}

\begin{frame}
    \frametitle{Functions}

    \begin{itemize}[<+->]
        \item Functions are just objects with two special properties: a context (scope) and the
            function code.
        \item Functions can be defined anywhere where an object can be defined and can be stored in
            variables.
        \item Functions can access all arguments passed to a function via the \texttt{arguments}
            variable.
        \item Functions can also have named parameters.
        \item Functions always return a value. If no return is explicitly specified, the function
            will return undefined.
        \item Functions invoked using the new command construct objects. The default return value is
            \texttt{this}.
    \end{itemize}
\end{frame}

\begin{frame}[fragile]
    \frametitle{Functions: Callback}
    Since JavaScript functions are objects, they can be passed just like other objects.

    \begin{minted}{javascript}
    function doStuff(callback) {
        // do a bunch of processing
        var x = 3;
        console.log('in doStuff');
        callback(x);
    }

    doStuff(function(x) {
        console.log(x * 3);
    });
    \end{minted}

    \textbf{Output:}
    \begin{minted}{text}
    in doStuff
    9
    \end{minted}
\end{frame}

\begin{frame}[fragile]
    \frametitle{Scope I}

    There are two scopes in JavaScript: global and function.\footnotemark[1]\\

    Variables are \textit{hoisted} to the top of the function they are declared in. Thus, the
    following is entirely valid.

    \begin{minted}{javascript}
    function scopeEx() {
        b = 5;
        console.log(b); // logs 5
        var b = 3
        console.log(b); // logs 3
    }
    \end{minted}

    This is confusing. It is recommended that you declare all of your variables at the top of your
    functions (one exception to this rule is counter variables).

    \footnotetext[1]{In ES6, variables declared with \texttt{let} are actually block scope.}
\end{frame}

\begin{frame}[fragile]
    \frametitle{Scope II}

    Variables declared outside of a function are automatically in the global scope.\\

    Variables declared within a function \textit{without} the \texttt{var} keyword are also in the
    global scope.

    \begin{minted}{javascript}
    var a = 2;
    (function() {
        b = 3
        var c = 5;
    })(); // this creates and invokes the function
          // immediately

    console.log(a); // logs 2
    console.log(b); // logs 3
    console.log(c); // error since c is undefined
                    // in global scope
    \end{minted}
\end{frame}

\begin{frame}
    \frametitle{Global Abatement}
    Because your code could coexist with other people's code, on the same HTML page, it is
    recommended that you reduce your \textit{global footprint} by creating only a few global objects
    and then putting all assets into that object.\\

    Since you can add properties to objects at will, you can still split your code into multiple
    files.
\end{frame}

\begin{frame}[fragile]
    \frametitle{Functions: Closure}

    The fact that you can declare a function from within a function allows you to simulate private
    variables.\\

    \begin{minted}{javascript}
    var Dog = function(name) {
        var gender = 'male';
        this.name = name;
        this.isBoy = function () {
            return gender == 'male';
        };
    };

    var myDog = new Dog('Sebastian');
    console.log(myDog.gender);  // logs undefined
    console.log(myDog.name);    // logs 'Sebastian'
    console.log(myDog.isBoy()); // logs true
    \end{minted}
\end{frame}

\begin{frame}
    \frametitle{Syntax: Types\footnotemark[1]}

    JavaScript has six primitive types:

    \begin{itemize}
        \item Boolean
        \item Null
        \item Undefined (yes, this is a type)
        \item Number (can be a number between $-(2^{53} - 1)$ and $2^{53} - 1$, \texttt{NaN},
            \texttt{-Infinity}, or \texttt{Infinity}). There is only one number type, a 64-bit
            floating point number.
        \item String (single or double quotes declares a string literal\footnotemark[2])
        \item Symbol (new in ECMAScript 6)
    \end{itemize}

    \footnotetext[1]{Info on this slide from:
    \url{https://developer.mozilla.org/en-US/docs/Web/JavaScript/Data_structures}}
    \footnotetext[2]{Single quotes are recommended by Douglas Crockford because HTML normally uses
        double quotes and to avoid conflicts when manipulating DOM objects, single quotes should be
    used.}
\end{frame}

\begin{frame}
    \frametitle{Syntax: Variables}

    JavaScript is an \textbf{untyped} language. I don't know what that means and I don't
    think that Brendan did either when he wrote the language.\\

    Variables are declared using the \texttt{var} keyword\footnotemark[1]. \\

    \textbf{Examples:}

    \begin{itemize}
        \item \texttt{var name;} - creates variable \texttt{name} of type \texttt{undefined}.
        \item \texttt{var name = `Sumner';} - string literal
        \item \texttt{var age = 18;} - declaring a number literal
        \item \texttt{var hasFriends = false;} - declaring a boolean
        \item \texttt{var significantOther = null;}
    \end{itemize}

    \footnotetext[1]{Sometimes you don't need to use \texttt{var} as I have described above.}
\end{frame}

\begin{frame}[fragile]
    \frametitle{Syntax: Arrays}

    JavaScript arrays are basically vectors.

    \begin{minted}{javascript}
    var arr = [1, 'a', {}, [], 4, true];
    arr[0] = 'not a number';
    arr.push('this is basically a vector');
    console.log(arr);
    \end{minted}

    \textbf{Output:}
    \begin{minted}{text}
    [ 'not a number', 'a', {}, [], true, 'this is basically a vector' ]
    \end{minted}

    \textit{Note that the elements of an array do not have to be the same type.}

\end{frame}

\begin{frame}[fragile]
    \frametitle{Syntax: Objects}

    \begin{minted}{javascript}
    var myObj = { // this is an object literal
        a: 3,
        'b': 'JavaScript',
        'is-awesome?': true,
        doSomething: function () {
            console.log(this.a); // 3
            console.log(a); // error
        }, // trailing commas are allowed
    };
    myObj.doSomething();
    console.log(myObj.b, myObj['is-awesome?']);
    \end{minted}

    \textbf{Output:}
    \begin{minted}{text}
    3
    error: a is undefined
    JavaScript true
    \end{minted}

\end{frame}

\begin{frame}[fragile]
    \frametitle{Syntax: Control Statements}

    \begin{minted}{javascript}
    // if statement syntax is identical to C++
    if (condition) {
    } else if (condition) {
    } else {
    }

    // ternary syntax is just like C++
    var a = condition ? val_if_true : val_if_false;

    for (initializer; condition; incrementor) {
        // for loop syntax is identical
    }

    // iterated for loop
    for (var prop in obj) { // don't use for arrays where order matters
        if (obj.hasOwnProperty(prop)) { // don't iterate over prototype
            // process the stuff
        }
    }
    \end{minted}
\end{frame}

\begin{frame}
    \frametitle{Pitfalls: Truthy, Falsy and \texttt{==} vs \texttt{===}}

    JavaScript has the notion of being \textit{truthy} and \textit{falsy}.\\

    The following values are always falsy: \texttt{false, 0, "", null, undefined, NaN}. \\

    Do not expect all falsy values to be equal to each other (\texttt{false == null} is false). \\

    JavaScript has two equality operators:
    \begin{itemize}
        \item \texttt{==} compares without checking variable type. This will cast then compare.
        \item \texttt{===} compares and checks variable type.
    \end{itemize}
\end{frame}

\begin{frame}
    \frametitle{Additional Resources}

    A lot of this presentation was based off of \textit{JavaScript: The Good Parts} by Douglas
    Crockford. This is an essential read for anyone interested in learning JavaScript for anything
    more than writing a few simple scripts.\\

    MDN is the best resource for JavaScript documentation
    (\url{https://developer.mozilla.org/en-US/}). \\

    \textbf{JSHint} (\url{http://jshint.com/about/}) is a tool which checks JavaScript syntax and
    helps prevent bugs in your code. JSHint has plugins for most IDEs and text editors. Here's a SO
    article on the Vim plugin: \url{http://stackoverflow.com/questions/473478/vim-jslint/5893447}\\

\end{frame}

\begin{frame}
    \frametitle{Additional Resources: Libraries}

    There are \textbf{lots} of JavaScript libraries. One of the most widely used is jQuery
    (\url{http://jquery.com/}). It has good documentation and is really good for DOM
    manipulation.\\

    % TODO: Sam fill in details on other libraries
\end{frame}

\begin{frame}
    \frametitle{DOM Manipulation}
    The \textit{Document Object Model} is an API used by JavaScript to interact with the elements of
    an HTML document.\footnotemark[1]

    % TODO: Sam: fill in details using jQuery

    \footnotetext[1]{\url{https://en.wikipedia.org/wiki/Document_Object_Model}}
\end{frame}

\begin{frame}
    \frametitle{Canvas Manipulation}

    % TODO: SAM: Fill in details
\end{frame}

\begin{frame}
    \frametitle{Sources}

    I relived heavily on \textit{JavaScript the Good Parts} by Douglas Crockford in preparing this
    presentation. In fact, almost every slide contains some information I got from that book.
\end{frame}

\end{document}
